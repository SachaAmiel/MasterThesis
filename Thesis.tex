\documentclass[a4paper,11pt]{article}
%\usepackage[utf8]{inputenc}

\usepackage{amsmath, amsfonts, amsthm, bbm}
\usepackage{dsfont} % to do mathbb{1}
\usepackage{graphicx} 
\usepackage{bmpsize}
\usepackage{tikz}
\usepackage{braket}
\usetikzlibrary{arrows,calc,patterns,decorations.markings,decorations.pathreplacing,plotmarks,shapes.arrows,decorations.pathmorphing,backgrounds}
\tikzset{snake it/.style={decorate, decoration=snake}}
\usepackage[all]{xy}

\usepackage{bm}
\usepackage{hyperref}
\usepackage{float}
\usepackage{titling}
\usepackage{caption}
\usepackage{subcaption}
\usepackage{tikz-cd}

\numberwithin{equation}{section}
\theoremstyle{definition}
\newtheorem{definition}{Definition}
\newtheorem{theorem}{Theorem}
\newtheorem{lemma}{Lemma}
\newtheorem{prop}{Proposition}
\newtheorem{comment}{Comment}
\newcommand{\dbyd}[2]{\frac{\partial #1}{\partial #2}}
\newcommand{\vsig}{{\bm{\sigma}}}
\newcommand{\R}{\mathbb{R}}
\newcommand{\uone}{{u_1}}
\newcommand{\utwo}{{u_2}}
\newcommand{\chii}{\chi^{\vphantom{|}}}
\newcommand{\Tr}{\mathrm{Tr}}
\newcommand{\cH}{{\cal H}}
\newcommand{\NS}{{{\scriptstyle N\!S}}}
\newcommand{\tq}{{\tilde q}}
\renewcommand{\d}{{\mathrm{d}}}
\newcommand{\e}{{\mathrm{e}}}


\setlength{\parindent}{0pt}
\setlength{\parskip}{3pt}

\title{Quantum energy inequalities with curvature}
\author{Author: Sacha AMIEL \\
Student number: 23126576\\
\\ Supervisor: Dr. Eleni-Alexandra Kontou \\ Module Code: 7CCMTP50}
\date{2024/09}

\usepackage{geometry}
\geometry{a4paper, left=25mm, right=25mm, top=30mm, bottom=25mm}

\renewcommand{\baselinestretch}{1.2}

\begin{document}


\clearpage\maketitle
\thispagestyle{empty}
\begin{figure}[H]
    \centering
    \vspace{100mm}
    \includegraphics[width=0.2\columnwidth]{Template/kcl_logo.png}
\end{figure}

\newpage
\section*{Abstract}
This \color{red} text \color{black} will be, as explicitely stated in the title, about quantum ernergy inequalities with curvature. It will consist in a formal introduction to an axiomatic approach to quantum field theory, followed by a formal introduction to energy inequalities and will finish with quantum energy inequalities in curved spacetimes and of the knowledge they bring.

Quantum Energy Inequalities are covariant lower bounds on the energy of a given physics. They do not tell us much about the physics itself, nor about the spacetime the physics unfolds; but from the restrictions they impose on energy, one can (if following general relativity) impose restrictions on that spacetime. This can bring crucially insight in the field of quantum field theory in curved spaces, and therefore, potentially quantum gravity.

\section*{Informal introduction}
It is commonly known in physics that "energy should be positive", although the more one thinks about it, the less one is sure about it. In the life of a physics student, many formal statements are learned, "energy is defined up to a constant", "energy needs to be bounded from below", "energy is mass so it cannot be negative", "energy is the curvature of spacetime"... And overall, with time, one gets very flexible with what one can accept as an energy.

In this \color{red} text \color{black} we will take the approach of mathematical physics, striving to only work with rigorously well defined concept. This will lead to results that are less impressive but probably more meaningful. Indeed, the point of quantum energy inequalities is not to claim any result about quantum gravity, but simply to exclude some types of spacetime if one desires to build a physical model with both gravity and quantum field theory. 
\newpage
\tableofcontents
\newpage
\section{Defining Quantum Field Theory in Curved Spacetimes}
This section will make use (without defining them) of the notion of category theory. To see the formal definition \color{red} ref!!!!! \color{black}. We will use the symbol $\mathbb{M}_4$ to refer to Minkowski spacetime in 1+3 dimensions, and $\mathcal{M}$ will denote any spacetime with signature $\pm (-, +,+, ..., +)$.

\subsection{Axiomatic approach to Algebraic Quantum Field Theory}
\subsubsection{Formalizing the intuitive idea f a QFT}
\begin{definition}[Quantum Mechanics - \emph{Dirac style}]$\quad$

A quantum Mechanics is the data $(\mathbb{H},\mathcal{T})$ of a Hilbert space $\mathbb{H}$ and some \emph{Time evolution principle} $\mathcal{T}$. In pure theory $\mathcal{T}$ could be a labeling of \emph{everything} with respect to time, but in practice, it will usually be given by some time-evolution equation on either the vectors of $\mathbb{H}$, most operators of $\mathbb{H}$, or in some very specific cases, both...\\
An example of vector evolution would be, given a Hermitian operator $H\in\mathcal{B}(\mathbb{H})$, the well known Schrödinger-picture time-evolution:
$$\boxed{\mathcal{T}_{H,\mathrm{Sch}}: \Bigg\{
\begin{matrix}
\mathbb{H} & \to & \mathcal{C}^\infty(\mathbb{R},\mathbb{H})\\
\Psi & \mapsto & \Big(t\mapsto \e^{\mathfrak{i}\hbar H}\Psi\Big)
\end{matrix}}$$
An example of an operator time evolution would be the well known Heisenberg-picture time-evolution (which, if taken with the same $H$ is \emph{physically equivalent}):
$$\boxed{\mathcal{T}_{H,\mathrm{Hei}}: \Bigg\{
\begin{matrix}
\mathcal{B}(\mathbb{H}) & \to & \mathcal{C}^\infty\big(\mathbb{R},\mathcal{B}(\mathbb{H})\big)\\
X & \mapsto & \Big(t\mapsto \e^{-\mathfrak{i}\hbar H}X\e^{\mathfrak{i}\hbar H}\Big)
\end{matrix}}$$
But some more whimsical time-evolutions are used, for instance, in evolved methods of perturbation theory in molecular physics (drawing from a mixture of the previously mentioned ones)... In a way, once can consider the Dirac equation or the Klein-Gordon equations as being other time-evolutions, also defining a quantum mechanics (though in practice, we want to see them as part of a Quantum Field theory, which is something we will define next).

Note that the definition can often extend all the sets $\mathcal{B}(\mathbb{H})$ to $\mathcal{L}(\mathbb{H})$, i.e. disregard the requirement of boundedness. Although this makes things much harder to define mathematically (exponentials of operators suddenly require a lot more work in their definition), the idea remains exactly the same.
\end{definition}

\begin{definition}[States - Observables - and Physically Identical QMs] \emph{vocabulary}


"States" are the set $\mathbb{C}\mathcal{P}_\mathbb{H}$, i.e. the projective space of the Hilbert space of the quantum mechanics. Recall that the projection of a vector space is the set of all equivalence classes $\{\lambda \Psi, \lambda\in\mathbb{C}\}$ for all vectors $\Psi$ in $\mathbb{H}$. In practice, one often confuses (non-zero) vectors of $\mathbb{H}$ with their projection and just calls them states, just keeping in mind that they are identical up to change of norm and phase.

"Observables" are self hermitian operators of $\mathbb{H}$. Or a sub-algebra of it (depending on the context). The physically observable information of a quantum mechanics is the spectrum\footnote{\color{red}Should I explain how? Meaning, should I explain how to construct a probability distribution?\color{black}} of these operators. Thus, if two QM yield the same spectra up to an isomorphysim of QMs (understand by that an isomorphysm of Hilbert spaces AND of time evolution) we say that they are descriptions of the same physical systems, i.e. the same.

As mentioned previously, QMs in the Schrodinger picture and the Heisenberg picture are physically the same when using the same $H$. More trivial cases of two quantum mechanics being the same are for instance when we keep the same time-evolution but replace the Hilbert Space by an isomorphic one (for instance, switching $\mathcal{S}(\mathbb{R}^3)$ with itself using the Fourier Transform, which is called switching from "momentum representation" to "position representation" is a pretty stander example).    
\end{definition}

\begin{definition}[Quantum Field Theory] \emph{informal definition}
    The broad idea of a quantum field theory (QFT) is that it is sort of a quantum mechanics, but that is is covariant and describes Fields...
    To this day, there is not yet a clear consensus on any proper axiomatic definition; but next we define an \emph{Algebraic Quantum Field Theory} (AQFT), which does fit in this fuzzy category, and then we will just assume all QFTs to be AQFTs.
\end{definition}
\begin{definition}[C*-Algebra]
    A C*-Algebra is a tuple $(\mathcal{A}, ||\cdot||)$ where:
    \begin{itemize}
        \item $\mathcal{A}$ is a unital *-Algebra over $\mathbb{C}$;
        \item $||\cdot||$ is a norm over $\mathcal{A}$ as a complex vector space;
        \item $\forall (A,B)\in\mathcal{A}\;\; ||AB||\leq||A||\cdot||B||$
        \item $\forall A\in\mathcal{A}\;\; ||A^*\cdot A||=||A^*||\cdot||A||=||A||^2$
        \item $\mathcal{A}$ is \emph{complete} in the $||\cdot||$-metric topology
    \end{itemize} 
For simplicity, we will always assume that $||\mathds{1}||=1$... That's just the redefinition $||\cdot||\mapsto\frac{||\,\cdot\,||}{||\mathds{1}||}$, so nothing to overthink there...
\end{definition}

\begin{comment}
\label{Com:C*&Bound}
Notice that $\mathcal{B}(\mathbb{H})$ forms a C*-Algebra, using duality ($\dag$) as $*$-operator and opperator norm ($||A||:=\mathrm{Sup} \{\frac{||A\Psi||}{||\Psi||},\Psi \in \mathbb{H}\}$) as algebra norm.
\end{comment}

\subsubsection{The axioms of AQFT}
The few following definitions are taken from \cite{AQFT_Intro}. They can also be found in \cite{pAQFT}, but that last one goes a bit further, as we will see in a moment.
\begin{definition}[AQFT]
    An AQFT is a tuple $\big(\mathcal{A}(\mathbb{M}_4), \mathcal{A}(\cdot)\big)$. Where $\mathcal{A}(\mathbb{M}_4)$ is an C*-Algebra and $\mathcal{A}(\cdot)$ is a functor from the category $\mathcal{O}(\mathbb{M}_4)$ of open causally convex subsets of $\mathbb{M}_4$ to that of sub-C*-Algebras of $\mathcal{A}(\mathbb{M}_4)$ obeying the following requirements:
    \begin{enumerate}
        \item \textbf{Locality}:  $\big\{\mathcal{A}(O), O \in \mathcal{O}(\mathbb{M}_4) \; \mathrm{with}\; O\; \mathrm{bounded}\big\}$ generates $\mathcal{A}(\mathbb{M}_4)$
        \item \textbf{Isotony}: $\forall (O_1, O_2) \in \mathcal{O}(\mathbb{M}_4)^2 \quad O_1\subset O_2 \implies \mathcal{A}(O_1)\subset\mathcal{A}(O_2)$
        \item \textbf{Causality}: $\forall (O_1, O_2) \in \mathcal{O}(\mathbb{M}_4)^2 \quad O_1 \; \mathrm{and}\; O_2\; \mathrm{causaly}\;\mathrm{disjoint}\; \implies \big[\mathcal{A}(O_1),\mathcal{A}(O_2)\big]=0$
        \item \textbf{Covariance}: there exists a functor $\alpha$ from the category of the (identity connected) Poincaré transformations of $\mathcal{O}(\mathbb{M}_4)$ to that of the sub-C*-Algebras of $\mathcal{A}(\mathbb{M}_4)$.
        \item \textbf{Dynamics}: $\forall (O_1, O_2) \in \mathcal{O}(\mathbb{M}_4)^2 \;\mathrm{with} \;O_1 \subset O_2$, if $O_1$ contains a set met exactly once by every extensible timelike curve in $O_2$, then $\mathcal{A}(O_1)=\mathcal{A}(O_2)$
    \end{enumerate}
Note that the true definition of an AQFT actually doesn't use C*-Algebras but simply *-Algebras. But since dropping the C* requirement makes things a lot more complicated (it's basically like disregarding any topological concerns) when it simply doesn't make false many important results, most researchers in the field focus on C*-AQFT and we will simply ignore the other ones.
\end{definition}

\begin{definition}[States - Observables] \emph{vocabulary of QFT}

"States" are elements $\omega: \mathcal{A}(\mathbb{M}_4)\to\mathbb{C}$ obeying the following properties:
\begin{itemize}
    \item $\forall \lambda\in\mathbb{C}, \forall(A,B)\in\mathcal{A}(\mathbb{M}_4)^2\quad \omega(\lambda A+B)=\lambda \omega(A)+\omega(B)$
    \item $\forall A \in \mathcal{A}(\mathbb{M}_4)\quad \omega(A^*\, A)\in \mathbb{R}_+$
    \item $\omega(\mathds{1})=1$
\end{itemize}

"Observables" are self adjoint elements of $\mathcal{A}(\mathbb{M}_4)$, i.e. the elements of: $\{A \in\mathcal{A}(\mathbb{M}_4):A=A^*\}$.
\end{definition}

\subsubsection{Relating AQFT to the intuitive ideas of QFT}
\begin{comment}
\label{Com:SymDiracWeyl}
    Recall that, from comment \ref{Com:C*&Bound}, given any Hilbert space $\mathbb{H}$, the bounded operators of it $\mathcal{B}(\mathbb{H})$ forms a C* Algebra.\\
    Notice that, linear forms $\frac{\bra{\Psi}\,\cdot\,\ket{\Psi}}{\braket{\Psi|\Psi}}$ gives us a "State" in the QFT sense, and that for all $\ket{\lambda\Psi}$ corresponding to the same state, that associated linear form will be identicall.\\
    All this, (associated to the fact that $\mathbb{M}_4$ can be foliated with respect to time) sort of hints us towards the idea that AQFT is indeed sort of what a QFT should be (i.e. a covariant QM).\\
    Now, it turns out that, intrinsically, a QFT is not a QM, for some subdle reasons. It would be more rigorous to say that every QFT \emph{contains several QMs}... Let us build up to this more formally:
\end{comment}

\begin{definition}[Representation of a C*-Algebra]$\quad$

    A representation of a C*-Algebra $\mathcal{A}$ is a triple $(\mathbb{H},\mathcal{D},\pi)$ where $\mathbb{H}$ is a Hilbert Space; $\mathcal{D}$ a dense subspace of it; and $\pi : \mathcal{A}\to\mathcal{B}(\mathcal{D})$ a morphism of C*-Algebra such that $\pi(\mathds{1}_{|\mathcal{A}})=\mathds{1}_{|\mathcal{D}}$\\
    Unsurprisingly, it will be called a \emph{faithfull} representation when $\mathrm{Ker}(\pi)=\{0\}$ and irreducible if without any invariant subspaces...
    \\ So from that, it becomes sort of obvious that any representation of the algrebra of a QFT, associated with a $\mathcal{T}$ to keep track of the time each subalgebra is from, does indeed lead to a proper QM.
\end{definition}

\begin{comment}
    It turns out that, given a state (a pure state) $\omega$ one can always build a representation that works as broadly teased in comment \ref{Com:SymDiracWeyl} and yeilds a faithfull irreducible representation, with a vector $\ket{\Omega}$ such that $\omega\simeq\bra{\Omega}\cdot\ket{\Omega}$.\\
    And that representation will be unique up to unitary equivalence (i.e. yeild exaclty the same QM). This is Theorem 10 in \cite{AQFT_Intro}. One can see this particular representation (called GNS representation) as a "canonical representation" of a QFT associated to a particular state.\\
    \color{red} I'm only mentionning it very broadly because it "looks obvious" and also I'm guessing physicists won't care about it... But perhaps I should make it more formal, and also do the actual proof (only the pure state case, to go faster...) After all, in spite of its obviousness, this canonical prepresentation is used everywhere in the article... \color{black}
\end{comment}

\begin{comment}
    In the case where any two such representations are unitarily equivalent, we see that the given QFT \emph{is} a quantum mechanics... But it turns out many QFTs do not have this property, in particular the algebra of operators of free feild theories do not have this property.\\
    This is characterized by the author as the intrinsic presence of quantum fields in a theory. All purely QFT statements of the algebraic formulation are written in terms of this study of inequivalent representations. In particular, the structure of all inequavalent reps, formulated with "superselection sectors" yeilds sort of an algebraic gauge theory...\\
    Of course, that part is a lot more handwayvy in the article, so I don't know about the details, a lot of outside theorems are quoted in it... I guess I'll learn it next...
\end{comment}


    \subsection{Separating the notion of a physics and its spacetime}
\subsubsection{An over-complicated definition of a Field}
        Recall that $\mathfrak{A}$ can be seen as a covariant functor from the spacetime to the Algebra of observables. Let $\bold{Loc}$ be the category of hyperbolic oriented and time oriented spacetimes related by respective embedding. Let $\bold{Vec}$ be the categroy of topological vector spaces (with continuous linear functions as morphisms). And let $\bold{Obs}$ be the category of locally convex C*Algebras (or just unital *Algebras in the context of perturbation theory, or topological Poisson Algebras in the classical case) with corresponding algebra morphisms as morphisms.\\
Let $\mathcal{D}$ be the functor from $\bold{Loc}$ to $\bold{Vec}$ that associates to every spacetime its comactly supported test functions.

\begin{definition}[Quantum Field]
    A Quantum Field is a natural transformation from the functor $\mathcal{D}$ to the functor $\mathfrak{A}$ of field theory composed with the forgetful functor from $\bold{Obs}$ to $\bold{Vec}$
\end{definition}

Although this is, in many way, an overkill definition, we see that this basically holds over any spacetime, this leads us to slightly weaken the axioms of AQFT (called pAQFT) as follows (see \cite{pAQFT} for more), it will, amongst other things, allows to include perturbation theory to the frame but also make it work on any spacetime:
\begin{itemize}
    \item A net to topological *-algebras with sequentially continuous product over a spacetime $\mathcal{M}$; $\mathcal{O}\mapsto\mathfrak{A}(\mathcal{O})$
    \item (Locality) $\mathfrak{A}(\mathcal{M})=\lim_{\mathcal{O}\subset\mathcal{M}}\mathfrak{A}(\mathcal{O})$
    \item all the section 2.3 (R's book) except the spectrum condition
    \item (Time-slicing axiom) Algebra of the neighborhood of a Cauchy surface of a Region is the algebra of the whole region.
    \item There is a covariant functor from the category of regions and their action under the groupof the space-time symettries and the algebra.
    \item, the spectrum condition can no longer be used on curved surfaces, at "translations" are not everywhere well defined. Instead, this will be fixed by the use of Hadamard states (\color{red} see Sec 5 to get it\color{black}).
\end{itemize}
\color{red} this part has been rushed a bit, as I will invariably need to re-do it in the final doc, to make it a progression of the form:
$$\mathrm{AQFT}_{\mathbb{M}_4}\to\mathrm{AQFT}_\mathcal{M}\to\mathrm{pAQFT}_\mathcal{M}$$
That, or, 
$$\mathrm{AQFT}_{\mathbb{M}_4}\to\mathrm{pAQFT}_{\mathbb{M}_4}\to\mathrm{pAQFT}_\mathcal{M}$$
I haven't decided...\\ \\
Also, I'm missin a subsubsection...\color{black}\\
A core part of AQFT is the fact that, once properly defined in terms of category, one can see there is sort of two independent objects: the \emph{spacetime}, and the \emph{QFT}. And the two are only linked by Locality and Covariance, both which are not so much a active requirement, but more of a coherence requirement.\\
This leads to the possibility to formalize this separation of the two notions as the two essential building blocks of an AQFT, and thus to the possibility of changing one without the other. This (purely mathematical fact) can be physically understood as follows one can consider a QFT regardless of the space-time it lives in, and formally swap that spacetime with another, thereby rigorously considering \emph{the same physics on a different spacetime}.\\
This is why AQFT is so suited for our topic: it allows to do any physics on any space, and thus to ask formally the question: what physics can (or cannot) work properly on a given spacetime.
\subsubsection{The understanding the abstract meaning of a Field and also its more down to earth meaning}
        Notice that although the previous definition was stated quite abstractly and abruptly, it also contain a very practical definition of fields (hidden in the category language): given a spacetime they are dual to test functions and have their end inside the algebra: i.e. they are in practice non-commutative distributions. \color{red} add a ref to distribution theory and probably add distribution to the annex, also I might want to add at this point in the thesis something like "distributions contain functions, dirac and other things... So they are exaclty what physicists are used to deal with in computation"... But I'm not sure, it both sound like something usefull to say, to calm any reader that panicked after seeing functors everywhere for a whole page... But might also feel insulting to any reader that understood it without any issue"\color{black}.
\subsection{From its axiomatization to its practical implementation}
\subsubsection{Computation-oriented depiction of a Field}
        % Axioms of Fields, and the smearing
\subsubsection{A first glance at Hadamard states}
\subsubsection{A first glance at Hadamard renormalization}
\section{Energy Inequalities and their use in General Relativity}
\subsection{Classical Energy Inequalities}
\subsubsection{The Philosophy of Energy Conditions}
\subsubsection{Zoology of Classical Energy Conditions}
\subsubsection{The Breaking of Classical Energy Condition}
\subsection{Quantum Energy Inequalites}
\subsubsection{General Formalism of Q-E-Cond}
\subsubsection{Down the Hadamard Rabbit Whole}
\subsubsection{Double Energy Conditions}
\section{Result stuff}


\section{Conclusion}
\newpage
\section{Physics Appendix}
    \subsection{Category, functor, natural transformation}
    For proper definition, see the math appendix or \cite{AlgLang}.
        \subsubsection{Category}
    Category theory is the study of structures. Formally, a category is a collection of objects and the links between them (links which we call the \emph{morphisms} of the category).\\
    Examples of famous categories are:
    \begin{itemize}
        \item Groups (known in physics as "symetries"), whose morphisms are group-preserving functions (i.e. $f(a.b)=f(a).f(b)$)
        \item *Algebras (known in physics as "observables"), whose morphisms are *Algebras preserving functions (i.e. $f(\lambda.a.b+c)=\lambda.f(a).f(b)+f(c)$ and $f(a^\dag)=f(a)^\dag$)
        \item Given a spacetime, the collection of all the charts on it forms a category whose morphism are all the switch to other nearby charts
        \item The category of Feynman diagrams where the morphism are relations between them (like a diagram being a sub-diagram of the other in the non-1PI case, or the replacement of a particular vertex with a different one).
    \end{itemize}
    The power of category-formulated statements is that they are not mathematical objects per say, but are \emph{about} mathematical objects, so they allow to take a useful step back.
    
    In cathegory theory, one usually draws relations in diagrams as follows, and calls them \emph{commutative} whenever all paths are the same. For example, the following diagram defines the multiplication in a $\mathbb{K}$-algebra $A$ (with $\mu$ the algebra product, $\eta$ is the unit element of the algebra).\\
    \begin{tikzcd}
    & \arrow[ld, "\mu\otimes\mathrm{id}"] A\otimes A \otimes A  \arrow[rd, "\mathrm{id}\otimes\mu"]\\
    A\otimes A \arrow[rd, "\mu"]& &\arrow[ld, "\mu"]A\otimes A& & A \otimes A \arrow[rd, "\mu"]& & A\otimes A \arrow[rd, "\mu"]\\
    & A & & & \mathbb{K}\otimes A \arrow[u, "\eta\otimes\mathrm{id}"] \arrow[equal]{r}& A & \arrow[u, "\mathrm{id}\otimes\eta"] A\otimes\mathbb{K} \arrow[equal]{r} & A
\end{tikzcd}
    

    \subsubsection{Functors}
    A functor is an object that sends a category to another cathegory. They are the most important tool in cathegory theory, as they allow formal study of the interaction between two apparently different structures.\\
    Here are examples of famous functors from physics.
    \begin{itemize}
        \item The functor from Lie Groups to Lie Algebras (it sends groups to their tangent planes, and group morphisms into lie Algebra morphisms)
        \item  The "exponantial functor", which sends any Lie Algebra to the associated compact Lie Group (it is the \emph{dual} functor to the previous).
        \item The Legrendre functor which sends the Euleur-Lagrange Picture of Classical Dynamics to the Hamilton-Jacobi Picture
        \item The Fourrier Functor which sends Classical Quantum Mechanics from the Position Representation to the Momentum Representation
        \item The Feynman rules functor that sends the category of feynman diagrams to their values and the relations between them into formulas.
    \end{itemize}
    Having a functor simply means being able to bring some of the interesting things of a structure into another. In our case, (QFT in curved spaces) a functor that we will see all the time is the Algebra Functor, which to any neighborhood of a spacetime associates all the observables accessible to an observer in that region, and therfore, we see that it is core in the \emph{covariance} requirement of QFT.

    \subsubsection{Natural Transformations}
    We do not use them much in this \color{red} text \color{black} but they are still quite important. A natural transformation is a \emph{morphism between functors} from a category to another. They allow to study functors themselves as a category (and thus, sort of "close" category into itself).\\
    Famous physical natural transformations are:
    \begin{itemize}
        \item The change of overall phase in quantum mechanics, whatever the categories (Hilbert Space, Obsevables...) and whatever the functor, one can slightly tweak it by a unitary transformation, without changing the physics.
        \item The affine re-parametrization of geodesics in space-time. Whatever functor one has in covariant physics that is based on a geodesic parameter, one can get a different one by re-parametrization of the geodesic parameter.
        \item Renormalization group flow which tweaks Feynman rules based on the renormalization parameters
    \end{itemize}
    
    
\newpage
\section{Mathematics Appendix}
    \subsection{Category, functor, natural transformation}
    The following definitions can be found in \cite{AlgLang}. For an understanding of their interpretation, see the Physics Appendix...
    \subsubsection{Category}
    A cathegory $\mathcal{C}$ is a tuple $(\mathrm{Obj},\mathrm{Hom},\circ)$ such that for all $A,B \in \mathrm{Obj}$ there is a class of entities called \emph{morphisms from} $A$ \emph{to} $B$ denoted $\mathrm{Hom}(A,B)$ following the following properties:
    \begin{itemize}
        \item $\forall A, B, C \in \mathrm{Obj}, \forall f\in \mathrm{Hom}(A,B), \forall g\in \mathrm{Hom}(B,C), \exists h \in \mathrm{Hom} (A,C) : h=f\circ g$
        \item $\forall A,B \in \mathrm{Obj} \; \exists \mathrm{id}_A \in \mathrm{Hom}(A,A): \forall f \in \mathrm{Hom}(A,B) \; \mathrm{id}_A\circ f=f$
        \item $\forall A,B \in \mathrm{Obj} \; \exists \mathrm{id}_B \in \mathrm{Hom}(B,B): \forall f \in \mathrm{Hom}(A,B) \; f\circ \mathrm{id}_B=f$
    \end{itemize}
    Equations in cathegory theory language are usually written in the form of a graph (called commutative diagram) where all possible composition paths are meant to be equal. For instance, bijectivity of a morphism is written as follows:\\
    A morphism $f \in \mathrm{Hom}(A,B)$ is called an iso-morphism when there exists $g \in \mathrm{Hom}(B,A)$ such that the following diagram commutes:
    $$\begin{tikzcd}[column sep=4em,row sep=4em,/tikz/column 2/.style={column sep=2em}]
\arrow[loop left]{l}{\mathrm{id}_A} A \arrow[r,bend left,"f"]
    & B \arrow[l,bend left,"g"] \arrow[loop right]{r}{\mathrm{id}_B}
\end{tikzcd}$$

\subsubsection{Functor}
    Let $\mathcal{C}$ and $\mathcal{D}$ be two categories. A functor $\mathfrak{F}$ is something that, to all elements $A\in\mathrm{Obj}_\mathcal{C}$, associates an element $\mathfrak{F}(A)\in\mathrm{Obj}_\mathcal{D}$ and to all $f \in \mathrm{Hom}(A,B)$ associates $\mathfrak{F}(f)\in\mathrm{Hom}(\mathfrak{F}(A),\mathfrak{F}(B))$ such that the two following properties hold:
    \begin{itemize}
        \item $\forall A \in \mathrm{Obj}_\mathcal{C}\; \mathfrak{F}(\mathrm{id}_A)=\mathrm{id}_{\mathfrak{F}(A)}$
        \item $\forall A,B,C \in \mathrm{Obj}_\mathcal{C}, \forall f \in \mathrm{Hom}(A,B), \forall g \in \mathrm{Hom}(B,C)\quad \mathfrak{F}(f)\circ_\mathcal{D}\mathfrak{F}(g)=\mathfrak{F}(f\circ_\mathcal{C}g)$
    \end{itemize}
\subsubsection{Natural Transformations}
A natural transformation is an objects linking functors together so as to make functors joined with natural transformations a category of its own. Formal axiomatization on any single natural transformation follows from this requirement.



\newpage
%%%%%%%%%%%%%%%%


\section{Energy Conditions by Sean Carrol \& Primer on Energy Conditions by Erik Curiel}
This section is based on chapter 4.6 of \cite{E_Cond}.

\color{red} There is not much to say... He also talks about the physical interpretation in the case of a perfect fluid, but that's pretty trivial... He also gives some inclusions (not all of them super clearly, but still, I could talk about that) but they are also pretty trivial, simply by looking at the definitions of things. The example of the perfect fluid shows that all the case where inclusions don't look trivial, they are simply false... So yeah, I'm ignoring all that...\color{black}

\begin{definition}[Energy Conditions]$\quad$

    Let $T^{\mu\nu}$ be the energy-momentum tensor of a (classical) field theory over a Lorentzian manifold $\mathcal{M}$ with signature $(+,-,...,-)$ whose vectors are denoted by the set $\mathbb{M}(\mathcal{M})$ and whose (respectively) time-like; space-like and null vectors are denoted by  $\mathbb{M}(\mathcal{M})^{(\mathrm{T})}$; $\mathbb{M}(\mathcal{M})^{(\mathrm{S})}$ and $\mathbb{M}(\mathcal{M})^{(\mathrm{N})}$; i.e. with $g$ the metric of $\mathcal{M}$ we mean:
    \begin{align*}
        \mathbb{M}(\mathcal{M})^{(\mathrm{T})}&:=\left\{l\in\mathbb{M}(\mathcal{M}):g_{\mu\nu}l^\mu l^\nu>0\right\}\\
        \mathbb{M}(\mathcal{M})^{(\mathrm{S})}&:=\left\{l\in\mathbb{M}(\mathcal{M}):g_{\mu\nu}l^\mu l^\nu<0\right\}\\
        \mathbb{M}(\mathcal{M})^{(\mathrm{N})}&:=\left\{l\in\mathbb{M}(\mathcal{M}):g_{\mu\nu}l^\mu l^\nu=0\right\}
    \end{align*}
We define the following acronyms (associated names in footnote\footnote{WEC - Weak Energy Condition\\NEC - Null Energy Condition\\DEC - Dominant Energy Condition\\NDEC - Null Dominant Energy Condition\\SEC - Strong Energy Condition}):
\begin{align*}
\textbf{WEC}: && \forall l \in \mathbb{M}(\mathcal{M})^{(\mathrm{T})} && T_{\mu\nu}l^\mu l^\nu\geq \;& 0\\
\textbf{NEC}: && \forall l \in \mathbb{M}(\mathcal{M})^{(\mathrm{N})} && T_{\mu\nu}l^\mu l^\nu\geq \;& 0\\
\textbf{DEC}: && \forall l \in \mathbb{M}(\mathcal{M})^{(\mathrm{T})} && T_{\mu\nu}l^\mu l^\nu\geq \;& 0 \quad \quad \wedge \quad \quad T^{\mu\nu}l_\mu \notin\mathbb{M}(\mathcal{M})^{\mathrm{S}}\\
\textbf{NDEC}: && \forall l \in \mathbb{M}(\mathcal{M})^{(\mathrm{N})} && T_{\mu\nu}l^\mu l^\nu\geq \;& 0 \quad \quad \wedge \quad \quad T^{\mu\nu}l_\mu \notin\mathbb{M}(\mathcal{M})^{\mathrm{S}}\\
\textbf{SEC}: && \forall l \in \mathbb{M}(\mathcal{M})^{(\mathrm{T}))} && T_{\mu\nu}l^\mu l^\nu\geq \;& \frac{1}{2}T^\mu_\mu l^\nu l_\nu
\end{align*}
\end{definition}

In \cite{Primer}, the author gives a lot more energy conditions, and in particular, rather than having these inequalities hold at every point, they can (weaker) hold once integrated over a portion of a geodesic, or (even weaker) a full geodesic.
There are also more whimsical energy conditions and a lot of interpretation on they philosophical meaning... But overall I don't think it's worth summarizing much...

\section{Renormalisation in perturbative AQFT by Kasia Rejner}
I will skip summarizing \cite{RejConf} as, at the moment, I have 12 pages of notes that are compressed to their fullest, so I prefer to wait and see what will be usefull once I have used it, rather than waist a lot of time typing it all in Late$\chi$...

\section{Hadamard renormalisation}
This section is based on \cite{HadRen}. But since they don't realy explain nor derive anything other than a bunch of formulas, I will just write the formulas I felt were important...

\begin{equation}
    T_{\mu\nu}=\frac{2}{\sqrt{-g}}\frac{\delta}{\delta g^{\mu\nu}}S[\phi,g_{\mu\nu}]
\end{equation}
\begin{equation}
    G^\mathrm{F}(x,x'):=\mathfrak{i}\bra{\psi}\mathcal{T}\phi(x)\phi(x')\ket{\psi}
\end{equation}
\begin{equation}
    (\square_x-m^2-\xi R) G^\mathrm{F}(x,x')=-\delta^D(x,x') \implies \delta^D(x,x')=[-g(x)]^{-^{\!1}\!/_{\!2}}\delta^D(x-x')
\end{equation}
let $2\sigma$ be the square of the geodesic distance; noticing $2\sigma=\sigma^{;\mu}\sigma_{;\mu}$
\begin{equation}
    \Delta(x,x'):=-[-g(x)]^{-^{\!1}\!/_{\!2}} \mathrm{det}\Big(-\sigma_{;\mu\nu'}(x,x')\Big)[-g(x')]^{-^{\!1}\!/_{\!2}}
\end{equation}
\begin{equation}
    \mathrm{notice}:\quad \square_x\sigma=D-2\Delta^{-^{\!1}\!/_{\!2}}\Delta^{^{1}\!/_{\!2}}\;_{;\mu}\sigma^{;\mu}\quad \mathrm{and}\quad \lim_{x'\to x}\Delta(x,x')=1
\end{equation}
The basis of Hadamard renormalization lies in objects being only generated by a few elements whose divergence one can simply control from there. \color{red} In spite of having red a lot about it, although I now know \emph{how} to do hadamard renormalization from these formulas, I have no clue where these formulas come from... So I still don't understand Hadamard renormalisation...\color{black}

In this next part, I changed the notation a bit so as to not have to have a single formula regardless of the dimension.
\begin{equation}
G^\mathrm{F}\!(x,x')=\frac{\mathfrak{i}\cdot\!\Bigl(\Gamma\!\left(\!^D\!/_{\!2}-1\right)\Bigl)^{1-\delta_{2,D}}}{2\cdot(2\pi)^{\!^D\!/_{\!2}}}\!\left(\!\!\frac{(1-\delta_{2,D})\cdot U(x,x')}{\Bigl(\sigma(x,x')+i\epsilon\Bigl)^{\!\!^D\!/_{\!2}-1}}+\mathds{1}_{2\cdot \mathbb{N}}(D) \cdot V(x,x')\cdot\mathrm{ln}\Bigl(\!\sigma(x,x')+i\epsilon\!\Bigl)+W(x,x')\!\!\right)
\end{equation}
\begin{align*}
    \exists (U_n, V_n, W_n) \in (?)^{D/2-1}\times(?)^\mathbb{N}\times(?)^\mathbb{N}:&\\
    U&(x,x')=\sum_{n=0}^{^D\!/_{\!2}-2}U_n(x,x')\sigma^n(x,x')\\
    V&(x,x')=\sum_{n\in\mathbb{N}}V_n(x,x')\sigma^n(x,x')\\
    W&(x,x')=\sum_{n\in\mathbb{N}}W_n(x,x')\sigma^n(x,x')
    \quad\quad\quad\quad\quad\quad\quad\quad\quad\quad
\end{align*}
i.e. one can tailor expand with respect to the square metric in the space of operators. But each operator can also be tailor expanded from one space variable.
\begin{equation}
    U_n(x,x')=u_n(x)+\sum_{p\in\mathbb{N}^*}\frac{(-1)^p}{p!}u_{n,p}(x,x')
\end{equation}
\begin{equation}
    V_n(x,x')=v_n(x)+\sum_{p\in\mathbb{N}^*}\frac{(-1)^p}{p!}v_{n,p}(x,x')
\end{equation}
for some reason, $W$ is not tailor expanded through it's sub-operators, but straight... \color{red}no idea why...\color{black}
\begin{equation}
    W(x,x')=w(x)+\sum_{p\in\mathbb{N}^*}\frac{(-1)^p}{p!}w_{p}(x,x')
\end{equation}
\begin{equation}
    \bra{\psi}T_{\mu\nu}(x)\ket{\psi}^\mathrm{ren} = \lim_{x'\to x} \mathcal{T}_{\mu\nu}(x,x')\left[-\mathfrak{i}G^\mathrm{F}(x,x')\right]
\end{equation}
\begin{equation}
    \bra{\psi}T_{\mu\nu}(x)\ket{\psi}^\mathrm{ren} = \frac{\Bigl(\Gamma\!\left(\!^D\!/_{\!2}-1\right)\Bigl)^{1-\delta_{2,D}}}{2\cdot(2\pi)^{\!^D\!/_{\!2}}}\left[\lim_{x'\to x}\mathcal{T}_{\mu\nu}(x,x')W(x,x')+\mathds{1}_{2\cdot \mathbb{N}}(D)\frac{D}{2}g_{\mu\nu}V_1\right]+\Theta_{\mu\nu}(x)
\end{equation}
There are also other formulas, but not sure how relevant they are...

\section{Energy conditions with QFT}
This one summarizes \cite{EleRev}, (or to be more precise, it summarizes the parts of \cite{EleRev} that cannot be found in other articles, as this review itself is very complete and reviews most of the points developed before... but also brings in a lot of new perspective, which is what I will try to focus on).

The form of a quantum energy condition is:
\begin{equation}
\boxed{\braket{\rho(f)}_\omega \geq - \braket{\mathcal{Q}(f)}_\omega}
\end{equation}
where $\rho:=T^\mathrm{ren}_{\;\;ab}t^at^b$ is the energy density (in some sense given by a choice of the $t$ used. $f$ is a test function (\color{red} or a distribution for some reason\color{black}) $\mathcal{Q}$ is a possibly unbounded operator. As it stands, it is a point-wize one. Integrating the inequality's $t$ over a geodesic of some type gives an integrated energy condition (complete geodesic, or full, depending on what we want).

Now, we call a condition trivial when:
\begin{equation} \label{triv}
    \exists c, c' \quad : \quad \braket{\rho(f)}_\omega \; \geq \; c + c'\cdot|\braket{\mathcal{Q}(f)}_\omega|
\end{equation}
Naturally, we call a condition non-trivial when $\lnot$(\ref{triv}) holds and of course, the here-above is the point-wize formulation if we at a $\forall t$ to the condition, and an integrated if $t$ is integrated over a geodesic.

Note that here, we work in the case of scalar theory. Extensions to fermions and spin $>0$ bosons are apparently non-trivial. \\

\underline{Example:}\\
in $\mathbb{M}_4$, with $:\!\!\rho\!: \;\; := T^\mathrm{ren}_{\;\;00} - \braket{T^\mathrm{ren}_{\;\;00}}_{\omega_0}$ ($\omega_0$ being $\mathbb{M}_4$'s vacum state and $\omega$ any state) and $\gamma$ the world-line of an inertial observer; Q-WEC is:
\begin{equation}
    \braket{:\!\!\rho\!: \circ \gamma}_\omega(f^2) \geq \frac{-1}{16\pi^2} \int_{-\infty}^{+\infty}\mathrm{d}t (f''(t))^2
\end{equation}\\

Important remark: in the case of interacting theories, energy conditions cannot be state independent.\\

One also sees some sort of a time-energy uncertainty relation pop out, using a self adjoint: $\sigma_a := \sqrt{\omega(a^2)-\omega(a)^2}$ the relation is:
\begin{equation}
    \sigma_H\sigma_a \; \geq \; \frac{1}{2} \bigg|\omega\Big([a,H]\Big)\bigg| \; = \; \frac{1}{2}\big|\partial_t \omega(a)\big|
\end{equation}
Also, to fix the negative energy problem, using $A$ for the area of the energy flux; $T$ for the time separation between events and with $\beta:=^{\;t_0}\!\!\!/_{\!T}$:
\begin{align*}
    \rho(t) := \frac{|\Delta E|}{A}\Big[-&\delta(t) + (1-\epsilon)\cdot\delta(1-T)\Big]\\
    \implies& \quad \int_{-\infty}^{+\infty}\mathrm{d}t \; \rho(t) \cdot f^2_{t_0}(t) \geq \frac{-C}{t_0^{\;4}} \quad \wedge \quad |\Delta E| \cdot T^3 \leq \frac{C\cdot A}{\beta^3\Big(f(0)^3 - (1+\epsilon) \cdot f(\beta^2)\Big)}
\end{align*}
\color{red} I don't truly understand these formulae yet, but I think I understand the point, so I think that's good for now, plus I might not actualy need it...

There also was a very interesting bit on a Feynman interpretations of paths applied to the problem of Energy conditions. I thought it would lead to a sort of Feynman "quantization" of the classical conditions to the quantum ones... But failed to see any "Feynman ideas" in the next section... Didn't mean I didn't understand the section, just not sure if it's more a philosophical idea of a factual method, anyway... The theorem in question is 4.1\color{black}\\

\begin{definition}[Trapped Surface] $\quad$\\
 a space-like surface of co-dimension 2 which has 2 null normals with negative expansion\footnote{\color{red}no idea what an "expansion" is...\color{black}} is called a trapped surface. \color{red}To read more about this, I need to visit [148] in the review, so far, I haven't found it, but I found a much shorter \hyperlink{https://arxiv.org/pdf/1107.1344}{paper} by the same author and of the same name, so it's probably only a matter of finding the long version.\color{black}
\end{definition}
This is apparently, the main tool in proving the famous singularity theorems. like this one:
\begin{theorem}
    If $\exists$ a compact achronal smooth trapped surface; if condition (27) holds (27 from the original review) and if $\mathcal{M}$ is globally hyperbolic with non-compact Cauchy surface; then $\mathcal{M}$ is futur null geodesic incomplete.\\\color{red}See \href{https://www.cambridge.org/core/books/large-scale-structure-of-spacetime/1E6B961EC9878EDDBBD6AC0AF031CC93}{[32]} and \href{https://arxiv.org/pdf/1410.5226}{[148]} for rigorous proof. In particular, I must loo at O'Neil's proof, as it devellops tools particularily usefull in the quantum case that is \href{https://ia902501.us.archive.org/27/items/mathematics_202103/%28Pure%20and%20Applied%20Mathematics%2C%20Volume%20103%29%20Barrett%20O%27Neill-Semi-Riemannian%20Geometry%20With%20Applications%20to%20Relativity-Academic%20Press%20%281983%29.pdf}{[35]}\color{black}.
\end{theorem}
Last probably usefull fact I found: null-geodesic conditions are not easily extended to the Quantum case, so that's a problem...



\newpage 
\begin{thebibliography}{99}

    \bibitem{AQFT_Intro}
    Fewster, C; Rejzner, K; \textit{Algebraic Quantum Field Theory -- an introduction}. Ar$\chi$iv \href{https://arxiv.org/abs/1904.04051}{DOI:1904.04051} (2019)

    \bibitem{E_Cond}
    Carrol, S; \textit{SPACETIME AND GEOMETRY -- An Introduction to General Relativity}. Cambridge University Press \href{https://www.cambridge.org/highereducation/books/spacetime-and-geometry/38EDABF9E2BADCE6FBCF2B22DC12BFFE#overview}{ISBN:978-1108488396} (2019)

    \bibitem{Primer}
    Curiel, E; \textit{A Primer on Energy Conditions}. Towards a Theory of Spacetime Theories. Einstein Studies, vol 13. Birkhäuser \href{https://arxiv.org/abs/1405.0403}{DOI:10.1007/978-1-4939-3210-8\_3} (2014)

    \bibitem{RejConf}
    Rejzner, K; \textit{Renormalization in Perturbative Algebraic Quantum Field Theory}. Masterclass and Workshop on "Higher Structures Emerging from Renormalisation"; \href{https://www.youtube.com/watch?v=3JVDJhFpuPY&ab_channel=ErwinSchr%C3%B6dingerInternationalInstituteforMathematicsandPhysics%28ESI%29}{4 talks at the \textit{Erwin Schrödinger International Institute for Mathematics and Physics} (ESI)} (2021)

    \bibitem{HadRen}
    Décanini, Y; Folacci, A; \textit{Hadamard Renormalization of the stress-energy tensor for a quantized scalar field in a general of arbitrary dimension}. Physical Review \href{https://arxiv.org/abs/gr-qc/0512118}{DOI:10.1103} (2008)

     \bibitem{EleRev}
     Kontou, E-A; Sanders, K; \textit{Energy conditions in general relativity and quantum field theory}. Ar$\chi$iv \href{https://arxiv.org/abs/2003.01815}{DOI:2003.01815} (2020)

    \bibitem{pAQFT}
    Rejzner, K.; \textit{Perturbative Algebraic Quantum Field Theory} Springer \href{https://link.springer.com/book/10.1007/978-3-319-25901-7}{ISBN:978-3-319-25899-7} (2016)

    \bibitem{AlgLang}
    Lang, S.; \textit{Algebra} Springer \href{https://link.springer.com/book/10.1007/978-1-4613-0041-0}{ISNB:978-1-4612-6551-1} (1993) \color{red} Specifically Chapter 1, section 11... Don't forget to mention it somewhere\color{black}

    \bibitem{GRWald}
    Wald, R.; \textit{General Relativity} U-Chicago Press \href{https://press.uchicago.edu/ucp/books/book/chicago/G/bo5952261.html}{ISNB:978-0-226-870335} (1984) \color{red}also, not sure I will need that one, but might be useful to quote all the basic stuff\color{black}
\end{thebibliography} 
\end{document}