\documentclass[a4paper,11pt]{article}
%\usepackage[utf8]{inputenc}

\usepackage{amsmath, amsfonts, amsthm, bbm}
\usepackage{dsfont} % to do mathbb{1}
\usepackage{graphicx} 
\usepackage{bmpsize}
\usepackage{tikz}
\usetikzlibrary{arrows,calc,patterns,decorations.markings,decorations.pathreplacing,plotmarks,shapes.arrows,decorations.pathmorphing,backgrounds}
\tikzset{snake it/.style={decorate, decoration=snake}}
\usepackage[all]{xy}

\usepackage{bm}
\usepackage{hyperref}
\usepackage{float}
\usepackage{titling}
\usepackage{caption}
\usepackage{subcaption}

\numberwithin{equation}{section}
\theoremstyle{definition}
\newtheorem{definition}{Definition}
\newtheorem{theorem}{Theorem}
\newtheorem{lemma}{Lemma}
\newtheorem{prop}{Proposition}
\newtheorem{comment}{Comment}
\newcommand{\dbyd}[2]{\frac{\partial #1}{\partial #2}}
\newcommand{\vsig}{{\bm{\sigma}}}
\newcommand{\R}{\mathbb{R}}
\newcommand{\uone}{{u_1}}
\newcommand{\utwo}{{u_2}}
\newcommand{\chii}{\chi^{\vphantom{|}}}
\newcommand{\Tr}{\mathrm{Tr}}
\newcommand{\cH}{{\cal H}}
\newcommand{\NS}{{{\scriptstyle N\!S}}}
\newcommand{\tq}{{\tilde q}}
\renewcommand{\d}{{\mathrm{d}}}
\newcommand{\e}{{\mathrm{e}}}


\setlength{\parindent}{0pt}
\setlength{\parskip}{3pt}

\title{Summary of the Articles I have red}
\author{Author: Sacha AMIEL \\
Student number: 23126576\\
\\ Supervisor: Dr. Eleni-Alexandra Kontou \\ Module Code: 7CCMTP50}
\date{2024/09}

\usepackage{geometry}
\geometry{a4paper, left=25mm, right=25mm, top=30mm, bottom=25mm}

\renewcommand{\baselinestretch}{1.2}

\begin{document}


\clearpage\maketitle
\thispagestyle{empty}
\begin{figure}[H]
    \centering
    \vspace{100mm}
    \includegraphics[width=0.2\columnwidth]{Template/kcl_logo.png}
\end{figure}

\newpage

\tableofcontents

\newpage
\section{Algebraic Quantum Fied Theory (AQFT)}
\label{Sec:AQFT}
\subsection{Very brief summary}
This article \cite{AQFT_Intro} explains the main concepts of AQFT. It is very "basic", in the sense that it is only describing AQFT on $\mathbb{M}_4$ (Minkowski space in 1+3 dimentions) and and doesn't even mention things such as \emph{renormalization}, which are things that I will need for the project.

\subsection{Let's juggle with definitions}
\begin{definition}[Quantum Mechanics - \emph{Dirac style}]$\quad$

A quantum Mechanics is the data $(\mathbb{H},\mathcal{T})$ of a Hilbert space $\mathbb{H}$ and some \emph{Time evolution principle} $\mathcal{T}$. In pure theory $\mathcal{T}$ could be a labeling of \emph{everything} with respect to time, but in practice, it will usually be given by some time-evolution equation on either the vectors of $\mathbb{H}$, most operators of $\mathbb{H}$, or in some very specific cases, both...\\
An example of vector evolution would be, given a Hermitian operator $H\in\mathcal{B}(\mathbb{H})$, the well known Schrödinger-picture time-evolution:
$$\boxed{\mathcal{T}_{H,\mathrm{Sch}}: \Bigg\{
\begin{matrix}
\mathbb{H} & \to & \mathcal{C}^\infty(\mathbb{R},\mathbb{H})\\
\Psi & \mapsto & \Big(t\mapsto \e^{\mathfrak{i}\hbar H}\Psi\Big)
\end{matrix}}$$
An example of an operator time evolution would be the well known Heisenberg-picture time-evolution (which, if taken with the same $H$ is \emph{physically equivalent}):
$$\boxed{\mathcal{T}_{H,\mathrm{Hei}}: \Bigg\{
\begin{matrix}
\mathcal{B}(\mathbb{H}) & \to & \mathcal{C}^\infty\big(\mathbb{R},\mathcal{B}(\mathbb{H})\big)\\
X & \mapsto & \Big(t\mapsto \e^{-\mathfrak{i}\hbar H}X\e^{\mathfrak{i}\hbar H}\Big)
\end{matrix}}$$
But some more whimsical time-evolutions are used, for instance, in evolved methods of perturbation theory in molecular physics (drawing from a mixture of the previously mentioned ones)... In a way, once can consider the Dirac equation or the Klein-Gordon equations as being other time-evolutions, also defining a quantum mechanics (though in practice, we want to see them as part of a Quantum Field theory, which is something we will define next).

Note that the definition can often extend all the sets $\mathcal{B}(\mathbb{H})$ to $\mathcal{L}(\mathbb{H})$, i.e. disregard the requirement of boundedness. Although this makes things much harder to define mathematically (exponentials of operators suddenly require a lot more work in their definition), the idea remains exactly the same.
\end{definition}

\begin{definition}[States - Observables - and Physically Identical QMs] \emph{vocabulary}


"States" are the set $\mathbb{C}\mathcal{P}_\mathbb{H}$, i.e. the projective space of the Hilbert space of the quantum mechanics. Recall that the projection of a vector space is the set of all equivalence classes $\{\lambda \Psi, \lambda\in\mathbb{C}\}$ for all vectors $\Psi$ in $\mathbb{H}$. In practice, one often confuses (non-zero) vectors of $\mathbb{H}$ with their projection and just calls them states, just keeping in mind that they are identical up to change of norm and phase.

"Observables" are self hermitian operators of $\mathbb{H}$. Or a sub-algebra of it (depending on the context). The physically observable information of a quantum mechanics is the spectrum\footnote{\color{red}Should I explain how? Meaning, should I explain how to construct a probability distribution?\color{black}} of these operators. Thus, if two QM yield the same spectra up to an isomorphysim of QMs (understand by that an isomorphysm of Hilbert spaces AND of time evolution) we say that are descriptions of the same physical systems, i.e. the same.

As mentioned previously, QMs in the Schrodinger picture and the Heisenberg picture are physically the same when using the same $H$. More trivial cases of two quantum mechanics being the same are for instance when we keep the same time-evolution but replace the Hilbert Space by an isomorphic one (for instance, switching $S(\mathbb{R}^3)$ with itself using the Fourier Transform, which is called switching from "momentum representation" to "position representation" is a pretty stander example).    
\end{definition}

\begin{definition}[Quantum Field Theory] \emph{informal definition}
    The broad idea of a quantum field theory (QFT) is that it is sort of a quantum mechanics, but that is is covariant and describes Fields...
    To this day, there is not yet a clear consensus on any proper axiomatic definition; but next we define an \emph{Algebraic Quantum Field Theory} (AQFT), which does fit in this fuzzy category, and then we will just assume all QFTs to be AQFTs.
\end{definition}
\begin{definition}[C*-Algebra]
    A C*-Algebra is a tuple $(\mathcal{A}, ||\cdot||)$ where:
    \begin{itemize}
        \item $\mathcal{A}$ is a unital *-Algebra over $\mathbb{C}$;
        \item $||\cdot||$ is a norm over $\mathcal{A}$ as a complex vector space;
        \item $\forall (A,B)\in\mathcal{A}\;\; ||AB||\leq||A||\cdot||B||$
        \item $\forall A\in\mathcal{A}\;\; ||A^*\cdot A||=||A^*||\cdot||A||=||A||^2$
        \item $\mathcal{A}$ is \emph{complete} in the $||\cdot||$-metric topology
    \end{itemize} 
For simplicity, we will always assume that $||\mathds{1}||=1$... That's just the redefinition $||\cdot||\mapsto\frac{||\cdot||}{||\mathds{1}||}$, so nothing to overthink there...
\end{definition}

\begin{comment}
Notice that $\mathcal{B}(\mathbb{H})$ forms a C*-Algebra, using duality ($\dag$) as $*$-operator and opperator norm ($||A||:=\mathrm{Sup} \{\frac{||A\Psi||}{||\Psi||},\Psi \in \mathbb{H}\}$) as algebra norm.
\end{comment}

\begin{definition}[AQFT]
    An AQFT is a tuple $\big(\mathcal{A}(\mathbb{M}_4), \mathcal{A}(\cdot)\big)$. Where $\mathcal{A}(\mathbb{M}_4)$ is an C*-Algebra and $\mathcal{A}(\cdot)$ is a functor from the category $\mathcal{O}(\mathbb{M}_4)$ of open causally convex subsets of $\mathbb{M}_4$ to that of sub-C*-Algebras of $\mathcal{A}(\mathbb{M}_4)$ obeying the following requirements:
    \begin{enumerate}
        \item \textbf{Locality}:  $\big\{\mathcal{A}(O), O \in \mathcal{O}(\mathbb{M}_4) \; \mathrm{with}\; O\; \mathrm{bounded}\big\}$ generates $\mathcal{A}(\mathbb{M}_4)$
        \item \textbf{Isotony}: $\forall (O_1, O_2) \in \mathcal{O}(\mathbb{M}_4)^2 \quad O_1\subset O_2 \implies \mathcal{A}(O_1)\subset\mathcal{A}(O_2)$
        \item \textbf{Causality}: $\forall (O_1, O_2) \in \mathcal{O}(\mathbb{M}_4)^2 \quad O_1 \; \mathrm{and}\; O_2\; \mathrm{causaly}\;\mathrm{disjoint}\; \implies \big[\mathcal{A}(O_1),\mathcal{A}(O_2)\big]=0$
        \item \textbf{Covariance}: there exists a functor $\alpha$ from the category of the (identity connected) Poincaré transformations of $\mathcal{O}(\mathbb{M}_4)$ to that of the sub-C*-Algebras of $\mathcal{A}(\mathbb{M}_4)$.
        \item \textbf{Dynamics}: $\forall (O_1, O_2) \in \mathcal{O}(\mathbb{M}_4)^2 \;\mathrm{with} O_1 \subset O_2$, if $O_1$ contains a set met exactly once by every extensible timelike curve in $O_2$, then $mathcal{A}(O_1)=\mathcal{A}(O_2)$
    \end{enumerate}
Note that the true definition of an AQFT actually doesn't use C*-Algebras but simply *-Algebras. But since dropping the C* requirement makes things a lot more complicated (it's basically like disregarding any topological concerns) when it simply doesn't make false many important results, most researchers in the field focus on C*-AQFT and we will simply ignore the other ones.
\end{definition}

\begin{definition}[Representation of a QFT]
    \color{red}to write\color{black}
\end{definition}
% after that, thm: a rep of a QFT yeilds a QM; then comments about non-unitaryly equivalent reps and guage theory; and that'll be it I'll leave the more complicated stuff for later once I know if it is relevant or not to Q-En-Eq


\newpage
\begin{thebibliography}{99}

    \bibitem{AQFT_Intro}
    \href{https://arxiv.org/abs/1904.04051}{Fewster, C; Rejzner, K; \textit{Algebraic Quantum Field Theory -- an introduction}. Ar$\chi$iv DOI:1904.04051 (2019)} 
        
\end{thebibliography}
\end{document}