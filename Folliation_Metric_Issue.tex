\documentclass[a4paper,11pt]{article}
\usepackage{hyperref}
\usepackage{amsmath, amsfonts, amsthm, bbm}
\usepackage{cancel, color}
\renewcommand{\d}{{\mathrm{d}}}

\begin{document}
This text is about an issue re-doing what was presented in \href{https://web.math.princeton.edu/~aretakis/columbiaGR.pdf}{this} article. To be precise, on page $53$, after having defined the null foliation, we have the following statements:
\begin{center}
    --------------
\end{center}
The metric $g$ with respect to the canonical coordinates is given by
\begin{equation}
    g=-2\Omega^2 \d u \d \underline{u} + (b^ib^j\cancel{g}_{ij})\d v \d \underline{u} - 2(b^i\cancel{g}_{ij})\d\theta^j\d\underline{u}+\cancel{g}_{ij}\d\theta^i\d\theta^j, \tag{4.3}
\end{equation}
where $\cancel{g}$ denotes the induced metric on the $2$-surfaces
$S_{u,\underline{u}}=C_u\cap \underline{C}_{\underline{u}}$. We immediately obtain
\begin{equation}
    \mathrm{det}(g)=-\Omega^2\cdot\mathrm{det}(\cancel{g}).\tag{4.4} \label{det}
\end{equation}
\begin{center}
    --------------
\end{center}

Now, disreguarding an obvious (at least, it seems to me) typo where a $u$ became a $v$, we see that the author claims that the metric is:
\begin{equation}\label{g}
    \boxed{g=(b^ib^j\cancel{g}_{ij}-2\Omega^2) \d u \d \underline{u} - 2b^i\cancel{g}_{ij}\d\theta^j\d\underline{u}+\cancel{g}_{ij}\d\theta^i\d\theta^j}
\end{equation}
And from that claim, follows equation (\ref{det}). So one would expect:
$$(\ref{g})\implies(\ref{det})$$
I have found (\ref{det})'s derivation to not be so immediate.\\
Reorganizing terms to do bloc matrix treatment, we have:
\begin{align*}
    g&= &+\left(\frac{1}{2}b^ib^j\cancel{g}_{ij}-\Omega^2\right)\d u \d \underline{u}\\
    &\quad\;+\left(\frac{1}{2}b^ib^j\cancel{g}_{ij}-\Omega^2\right)\d\underline{u}\d u &&-b^i\cancel{g}_{ij}\d\underline{u}\d\theta^j\\
    &&-b^i\cancel{g}_{ij}\d\theta^j\d\underline{u}\quad\quad\quad&+\cancel{g}_{ij}\d\theta^i\d\theta^j
\end{align*}
We can then use \href{https://en.wikipedia.org/wiki/Block_matrix#Determinant}{this} bloc-matrix theorem (with $D=\cancel{g}$):
\begin{equation}\label{bloc}
    \mathrm{det}\left(\begin{matrix}
        A & B\\
        C & D
    \end{matrix}\right) = \mathrm{det}(D)\cdot\mathrm{det}(A-BD^{-1}C)
\end{equation}
To get the following contradicting formula:
\begin{equation}
    \mathrm{det}(g)=-\left(\frac{1}{2}b^ib^j-\Omega^2\right)^2\cdot\mathrm{det}(\cancel{g})\label{mine}
\end{equation}
Which is incompatible with (\ref{det})...

Now, chances are (and they are quite high, as I am bad at computations) that (\ref{mine}) is incorrect. But judging by the shapes of both (\ref{det}) and (\ref{bloc}) the \emph{true} formula, whatever it is, is independent of the dimension, so let us explicit the computation in the case of a single $\theta$ dimension (i.e. in a 3 dimensional space-time).


\noindent \\ In 1+2D, $g$ would have the following shape:
\begin{equation}
    g_\mathrm{matrix}=\left(\begin{matrix}
        0 & \frac{b^2\cancel{g}}{2}-\Omega^2 & 0\\
        \frac{b^2\cancel{g}}{2}-\Omega^2& 0 & -b\cancel{g}\\
        0 & -b\cancel{g} & \cancel{g}
    \end{matrix}\right)
\end{equation}
where both $b$ and $\cancel{g}$ are now scalars...\\
We recall the well known formula:
\begin{equation}
    \mathrm{det}\left(\begin{matrix}
        a & b & c\\
        d & e & f\\
        g & h & i
    \end{matrix}\right)=aef +bfg+cdh - ceg - bdi - afh
\end{equation}
where (basically) one sums all products of diagonals over the periodicied 3x3 grid, with a corresponding sign.\\
This yields in particular:
\begin{equation*}
    \mathrm{det}\left(\begin{matrix}
        0 & a & 0\\
        a & 0 & b\\
        0 & b & c
    \end{matrix}\right)=-a^2c
\end{equation*}
from which is obtained instantly that:
\begin{equation} \label{mine3}
    \mathrm{det}(g_{3\mathrm{D}})=-\left(\frac{1}{2}b^2\cancel{g}-\Omega^2\right)^2\cancel{g} \quad\quad\quad\ne -\Omega^2\cancel{g} \tag{\ref{mine}.bis}
\end{equation}
i.e.
$$(\ref{g})\quad\quad\implies\quad\quad (\ref{mine}) \vee (\ref{mine3}) \quad\quad\implies\quad\quad \neg(\ref{det})$$
\\

Due to how nice (\ref{det}) looks, my guess is that it is correct (after all, $\Omega$ is defined to be the ``naturally extended" in the foliation...). So I believe (\ref{g}) is wrong. But either way, more details need to be looked at. $_\square$
\begin{center}
    ---------------
\end{center}
Trying to fix formula (\ref{g}) is hard. The $v\mapsto u$ assumtion I made is my first guess of what could be the cause with minimal change... But if we don't want to violate einstein summation conventions, the only possible change is $v \mapsto \underline{u}$ which doesn't work on its own. My current best guess is the following:
\begin{equation}
    g=-2\Omega^{\color{red}1\color{black}} \d u \d \underline{u} + b^ib^j\cancel{g}_{ij}\d\color{red}\underline{u}\color{black}\d\underline{u} - 2b^i\cancel{g}_{ij}\d\theta^j\d\underline{u}+\cancel{g}_{ij}\d\theta^i\d\theta^j \tag{¿fix?}
\end{equation}
I am pretty confident this works and have not found a more satisfying fix. As for re-derinving $g$'s expression from scratch, I'm not there yet...
\end{document}