\documentclass[a4paper,11pt]{article}
%\usepackage[utf8]{inputenc}

\usepackage{amsmath, amsfonts, amsthm, bbm}
\usepackage{dsfont} % to do mathbb{1}
\usepackage{graphicx} 
\usepackage{bmpsize}
\usepackage{tikz}
\usepackage{braket}
\usetikzlibrary{arrows,calc,patterns,decorations.markings,decorations.pathreplacing,plotmarks,shapes.arrows,decorations.pathmorphing,backgrounds}
\tikzset{snake it/.style={decorate, decoration=snake}}
\usepackage[all]{xy}

\usepackage{bm}
\usepackage{hyperref}
\usepackage{float}
\usepackage{titling}
\usepackage{caption}
\usepackage{subcaption}
\usepackage{tikz-cd}
\usepackage{cancel}

\numberwithin{equation}{section}
\theoremstyle{definition}
\newtheorem{definition}{Definition}
\newtheorem{theorem}{Theorem}
\newtheorem{lemma}{Lemma}
\newtheorem{prop}{Proposition}
\newtheorem{comment}{Comment}
\renewcommand{\d}{{\mathrm{d}}}
\newcommand{\e}{{\mathrm{e}}}
\newcommand{\R}{{\mathrm{Ric}}}

\newcommand{\EAK}[1]{\textcolor{red}{EAK: #1}}
\newcommand{\SA}[1]{\textcolor{red}{SA: #1}}

\setcounter{tocdepth}{2}
\setlength{\parskip}{3pt}

\title{Quantum energy inequalities with curvature}
\author{Author: Sacha AMIEL \\
Student number: 23126576\\
\\ Supervisor: Dr. Eleni-Alexandra Kontou \\ Module Code: 7CCMTP50}
\date{2024/09}

\usepackage{geometry}
\geometry{a4paper, left=25mm, right=25mm, top=30mm, bottom=25mm}

\renewcommand{\baselinestretch}{1.2}

\begin{document}
\SA{Okay, so I think I got it. All this time I though you needed a bound in terms of
\begin{align}
    |R^\rho_{\;\;\sigma\mu\nu}| &\leq R_\mathrm{max}&
    |\partial_\alpha R^\rho_{\;\;\sigma\mu\nu}| &\leq
    R_\mathrm{max}'&
    |\partial_\alpha\partial_\beta
    R^\rho_{\;\;\sigma\mu\nu}| & \leq R_\mathrm{max}''&
    \mathrm{ect}...
\end{align}
Because that is what was popping up in your articles all the time.
\\
But from the discussion we had last time, I guess these were circumstancial to the fact that your basis was ``physically relevant", leading to $X_{,\mu}=X_{;\mu}$...
\\
Thus, one can easily recycle my calculations to something in terms of covariant derivatives instead. Leading to:
}

\noindent
We will focus on the term $C_{\mu\nu}$'s computation. We mentioned previously (in section \ref{T_ren}) that :
\begin{equation}\tag{\ref{T_ren}}
    \braket{T^\mathrm{ren}_{\mu\nu}}:=\braket{T^\mathrm{fin}_{\mu\nu}}-Qg_{\mu\nu} + C_{\mu\nu}.
    \end{equation}
Let us be more precise. In \cite{QFTCurv}, we find that the coefficient is:
\begin{equation}
    -Qg_{\mu\nu}+C_{\mu\nu}=\alpha^{(1)}\mathcal{H}_{\mu\nu}+\beta^{(2)}\mathcal{H}_{\mu\nu}+\gamma\mathcal{H}_{\mu\nu},
\end{equation}
where:
\begin{align}
    ^{(1)}\mathcal{H}_{\mu\nu}:=& 2R_{;\mu;\nu}+ 2g_{\mu\nu}\square R - \frac{1}{2}g_{\mu\nu} R^2 + 2 R\cdot R_{\mu\nu}\notag;\\
    ^{(2)}\mathcal{H}_{\mu\nu}:=& R_{;\mu;\nu} - \frac{1}{2}g_{\mu\nu}R-\square R_{\mu\nu} -\frac{1}{2} R^{\alpha\beta}R_{\alpha\beta}+2R^{\alpha\beta}R_{\alpha\beta\mu\nu};\\
    \mathcal{H}_{\mu\nu}:=& 2R_{\mu\alpha\beta\gamma}R_\nu^{\;\alpha\beta\gamma}-4\square R_{\mu\nu} -\frac{1}{2}g_{\mu\nu}R^{\alpha\beta\gamma\delta}R_{\alpha\beta\gamma\delta} + 2 R_{;\mu;\nu} - 4 R_{\mu\alpha}R^{\alpha}_{\;\nu}+4R^{\alpha\beta}R_{\alpha\mu\beta\nu}\notag.
\end{align}
This yields:
\begin{align}
    C_{\mu\nu} &= 2\alpha R_{;\mu;\nu} + \beta R_{;\mu;\nu} - \beta\square R_{\mu\nu} +4\gamma \square R_{\mu\nu} + 2\gamma R_{;\mu;\nu} + \mathcal{O}(R^2)\notag\\
    &= (2\alpha + \beta + 2\gamma) R_{;\mu;\nu} +(4\gamma- \beta)\square R_{\mu\nu} + \mathcal{O}(R^2) \label{C=A+B}.
\end{align}

But as we saw in section \ref{TorsionWarning} (section \ref{TorsionWarning}), we need to remember that these expressions hold for coordinates in a \emph{basis}, whereas we wish to express them in our non-basis frame. To know exactly how this will end up looking, we will simply express the tensor $C$ without any reference to the basis $e_\mu$ and then, get its coefficients in the frame $(e_i,e_+,e_-)$.

Note that in this section, we will use the notation of $\nabla_\mu$ of the physicist, i.e. that of the coordinates. This is because \cite{QFTCurv} uses this one, so when we write $R_{;\mu;\nu}$ it means $\nabla_\mu \nabla_\nu R$ only in the physicist sense, and likewise for $\square:=\nabla_\rho \nabla^\rho$ in the physicist's way. We apologies for potentially caused confusion. Feel free to refer to comment \ref{NotationWarning} (in section \ref{NotationWarning}) for more details.

Now, the here-above expression of $C_{\mu\nu}$ is in a proper basis, one where the coordinate fields commute. So, to be sure that we do not make any mistakes when changing to our non-commutative frame, we will compute the mathematical tensor:
\begin{equation}
C : \Bigg\{\begin{matrix}
    T\mathcal{M}^{\otimes2} & \to & \mathbb{R}\\
    (x^\mu e_\mu, y^\nu e_\nu) & \mapsto & x^\mu y^\nu C_{\mu\nu}
\end{matrix}
\end{equation}
in a form that doesn't use any coordinate representation, then we will be able to evaluate it back in our frame, without any worries of what having non-commutative coordinate fields could cause being slightly different.\\
Then, we choose $l^\mu \partial_\mu$ to be along one of our null foliation coordinate $e_\pm$. \\
Leading to $l^\mu \partial_\mu = x_\pm e_\pm$ and the following integral:
\begin{equation}
    \int_\Sigma \d_\mathrm{vol}g^2 l^\mu l^\nu C_{\mu\nu} = \int_{\mathbb{R}\times \mathbb{R}\times S} \!\!\!\!\!\!\!\!\!\!\!\!\!\!\! \d_\mathrm{vol} \cancel{g}^2 \d x_\mp \d x_\mp' \d x_\pm \d x_\pm ' \;\cdot \; x_\pm x_\pm ' C_{\pm\pm}(x_i,x_\pm, x_\mp)\; ,
\end{equation}
with $C_{\pm\pm}:= C(e_\pm,e_\pm)$ of course. Proof of the measure's decomposition $\d_\mathrm{vol}g = \d x_\pm \d x_\mp \d_\mathrm{vol} \cancel{g}$ can be found in \cite{Art}. This also implies that $l^\mu \nabla_\mu$ will be $x_\pm \partial\pm$ when applied on to $\mathcal{FT}_{H_{(2)}}$.

\noindent So let us define (in a proper basis)
\begin{align}
    A_{\mu\nu} :=& \nabla_\mu\nabla_\nu R&
    B_{\mu\nu} :=& \nabla_\rho\nabla^\rho R_{\mu\nu}.
\end{align}
So as to have 
\begin{equation}
    C_{\mu\nu} = (2\alpha + \beta + 2\gamma) A_{\mu\nu} +(4\gamma- \beta) B_{\mu\nu}.
\end{equation}
And let us express their associated 2-tensors, using $X:=x^\mu e_\mu; Y:=y^\mu e_\mu;$ and $Z$ as dummy variables. We will use the notation $X\circ Y:= f \mapsto x^\mu \partial_\mu (y^\nu e_\nu f)$ for derivation along fields.
\begin{align}
    A(X,Y) :\!\!&= x^\mu y^\nu A_{\mu\nu}\notag\\
    :\!\!&= x^\mu y^\nu \nabla_\mu \nabla_\nu R\notag\\
    &= x^\mu y^\nu \nabla_\mu \partial_\nu R\notag\\
    &= x^\mu y^\nu \partial_\mu \partial_\nu R - x^\mu y^\nu \Gamma_{\mu\nu}^\lambda \partial_\lambda R\notag\\
    &= x^\mu \partial_\mu \left(y^\nu \partial_\nu R \right) - x^\mu \cdot(\partial_\mu y^\nu) \cdot (\partial_\nu R) - x^\mu y^\nu (\nabla_{e_\mu} \partial_\nu)R\notag\\
    &= x^\mu \partial_\mu \left(y^\nu \partial_\nu R \right) - \cancel{x^\mu \cdot(\partial_\mu y^\nu) \cdot (\partial_\nu R)} - x^\mu \big(\nabla_{e_\mu} (y^\nu \partial_\nu)\big)R + \cancel{x^\mu \cdot (\partial_\mu y^\nu) \cdot (\partial_\nu R)}\notag\\
    &= (X\circ Y - \nabla_X Y) R
\end{align}
Thus:
\begin{equation}
    A : \Bigg\{\begin{matrix}
        T\mathcal{M}^{\otimes 2} & \to & \mathbb{R}\\
        (X,Y) & \mapsto & \frac{1}{2}\big(X\circ Y + Y\circ X - \nabla_X Y - \nabla_Y X\big) R
    \end{matrix}
\end{equation}
\SA{I verified it, it turns out that having torsion (or non-commutative indices) doesn't change the $\nabla_\mu X_\nu$ formula. We only need to be very careful with indices and use:
$$\nabla_\mu X_\nu := \partial_\mu X_\nu - \Gamma_{\mu\nu}^\rho X_\rho \ne \partial_\mu X_\nu - \Gamma_{\nu\mu}^\rho X_\rho$$
The proof is identical to the normal case.}
\begin{align*}
    A_{\pm\pm} &\!\!:= A(e_\pm,e_\pm)
    \\&= \partial_\pm \partial_\pm R - (\nabla_{e_\pm}e_\pm) R\\
    &= \partial_\pm \nabla_\pm R- \Gamma_{\pm\pm}^\mu \partial_\mu R\\
    &= \partial_\pm R \nabla_\pm- \Gamma_{\pm\pm}^\mu \nabla_\mu R\\
    &= \nabla_\pm \nabla_\pm R
\end{align*}
\begin{equation}
    \boxed{A_{\pm\pm} = R_{;\pm;\pm}}
\end{equation}
So defining $\mathfrak{g}:= \sum_{\mu\nu} |g^{\mu\nu}|= 2 + \sum_{ij} |g^{ii}|$ we get:
\begin{equation}
    |A_{\pm\pm}| \leq \mathfrak{g} R_\mathrm{max}''
\end{equation}
\SA{Now, we could (like you suggested) just assume that all equations hold identically replacing $\mu$ by $\pm$, but we must not forget that $\partial_\mu\partial_\nu=\partial_\nu\partial_\mu$ in a normal basis, whereas, in our case, $\partial_\pm$ and stands for the vector field $e_\pm$ that doesn't commute with its opposite (index-wise) nor with the basis of the sub-manifold. So let me still do the painful computation.}
\\
First, let us acknowledge the fact that $B(X,Y)=\mathrm{tr}_g(Z\mapsto z^\rho z^\sigma x^\mu y^\mu \nabla_\rho \nabla_\sigma R_{\mu\nu})$. Which is already a bit scary. Next, we compute the tensorial (physics notation) expression
\SA{Let us skip the previously done work and just try to compute the mathematically abstract tensor $D$ that is in the thesis, only with great caution with respect to commutativity.}

\begin{equation}
    B:\bigg\{ \begin{matrix}
        T\mathcal{M}^{\otimes 2} & \to & \mathbb{R}\\
        (X,Y) & \mapsto & \mathrm{tr}_g \big(Z \mapsto D(X,Y,Z,Z)\big)
    \end{matrix}
\end{equation}

\begin{align}
    D(X,Y,Z,Z') 
    &=\R(\nabla_Z\nabla_{Z'} X ,Y )
    + \R( X ,\nabla_Z\nabla_{Z'} Y )
    + \R(\nabla_{\nabla_ZZ'} X , Y )
    + \R( X ,\nabla_{\nabla_ZZ'}Y )\notag\\
    & \quad + Z\big(Z'\big(\R(X ,Y )\big)\big)-(\nabla_Z Z')\R(X ,Y )
    -2 Z\big(\R(\nabla_{Z'} X , Y )\big)
    -2 Z\big(\R(X , \nabla_{Z'} Y )\big) \notag\\
    & \quad + 2 \R(\nabla_Z X , \nabla_{Z'} Y ) 
\end{align} 
This is a well defined tensor, let us evaluate it in \textit{some} basis $(e_\mu)$ not assuming commutativity (and denoting $\partial_\mu := e_\mu$ even when that might be confusing...)
\begin{align*}
    D_{\mu\nu\rho\sigma} 
    &=\R(\nabla_{e_\rho}\nabla_{e_\sigma} e_\mu ,e_\nu )
    + \R( e_\mu ,\nabla_{e_\rho}\nabla_{e_\sigma} e_\nu )
    + \R(\nabla_{\nabla_{e_\rho}e_\sigma} e_\mu , e_\nu )
    + \R( e_\mu ,\nabla_{\nabla_{e_\rho}e_\sigma}e_\nu ) \\
    & \quad + {e_\rho}\big(e_\sigma\big(\R(e_\mu ,e_\nu )\big)\big)-(\nabla_{e_\rho} e_\sigma)\R(e_\mu ,e_\nu )
    -2 {e_\rho}\big(\R(\nabla_{e_\sigma} e_\mu , e_\nu )\big)
    -2 {e_\rho}\big(\R(e_\mu , \nabla_{e_\sigma} e_\nu )\big)  \\
    & \quad + 2 \R(\nabla_{e_\rho} e_\mu , \nabla_{e_\sigma} e_\nu ) \\
    &=\R(\nabla_{e_\rho}\left(\Gamma_{\sigma\mu}^\alpha e_\alpha \right),e_\nu )
    + \R( e_\mu ,\nabla_{e_\rho}\left(\Gamma_{\sigma\nu}^\alpha e_\alpha \right) )
    + \Gamma_{\rho\sigma}^\alpha \Big(
    \R(\nabla_{e_\alpha} e_\mu , e_\nu )
    + \R( e_\mu ,\nabla_{e_\alpha}e_\nu )\Big) \\
    & \quad + \partial_\rho \partial_\sigma R_{\mu\nu} -(\nabla_{e_\rho} e_\sigma) R_{\mu\nu}
    -2 \partial_\rho \big(\R(\nabla_{e_\sigma} e_\mu , e_\nu )\big)
    -2 \partial_\rho\big(\R(e_\mu , \nabla_{e_\sigma} e_\nu )\big)  \\
    & \quad + 2 \Gamma_{\rho\mu}^\alpha \Gamma_{\sigma\nu}^\beta R_{\alpha\beta}\\
    &=\Gamma_{\sigma\mu}^\alpha\R(\nabla_{e_\rho}e_\alpha ,e_\nu )
    + \Gamma_{\sigma\nu}^\alpha\R( e_\mu ,\nabla_{e_\rho}e_\alpha )
    + \partial_\rho \Gamma_{\sigma\mu}^\alpha R_{\alpha\nu}
    + \partial_\rho\Gamma_{\sigma\nu}^\alpha R_{\mu\alpha} \\
    &\quad 
    + \Gamma_{\rho\sigma}^\alpha \Big(
    \R(\nabla_{e_\alpha} e_\mu , e_\nu )
    + \R( e_\mu ,\nabla_{e_\alpha}e_\nu )\Big)
    - \partial_\rho \big(\R(\nabla_{e_\sigma} e_\mu , e_\nu )\big)
    - \partial_\rho\big(\R(e_\mu , \nabla_{e_\sigma} e_\nu )\big)  \\
    & \quad + \partial_\rho \Big( \partial_\sigma R_{\mu\nu} 
    - \R(\nabla_{e_\sigma} e_\mu , e_\nu ) 
    - \R(e_\mu , \nabla_{e_\sigma} e_\nu )\Big)
    -(\nabla_{e_\rho} e_\sigma) R_{\mu\nu}  
    + 2 \Gamma_{\rho\mu}^\alpha \Gamma_{\sigma\nu}^\beta R_{\alpha\beta}\\
    &=\Gamma_{\sigma\mu}^\alpha\Gamma_{\rho\alpha}^\beta R_{\beta\nu}
    + \Gamma_{\sigma\nu}^\alpha\Gamma_{\rho\alpha}^\beta R_{\mu\beta}
    + \partial_\rho \Gamma_{\sigma\mu}^\alpha R_{\alpha\nu}
    + \partial_\rho\Gamma_{\sigma\nu}^\alpha R_{\mu\alpha} \\
    &\quad 
    + \Gamma_{\rho\sigma}^\alpha \Big(
    \Gamma_{\alpha\mu}^\beta R_{\beta\nu}
    + \Gamma_{\alpha\nu}^\beta R_{\mu\beta}\Big)
    - \partial_\rho \big(\Gamma_{\sigma\mu}^\alpha R_{\alpha\nu}\big)
    - \partial_\rho\big(\Gamma_{\sigma\nu}^\alpha R_{\mu\alpha}\big)  \\
    & \quad + \partial_\rho \Big( \partial_\sigma R_{\mu\nu} 
    - \Gamma_{\sigma\mu}^\alpha R_{\alpha\nu}
    - \Gamma_{\sigma\nu}^\alpha R_{\mu\alpha} \Big)
    -\Gamma_{\rho\sigma}^\alpha \partial_\alpha R_{\mu\nu}  
    + 2 \Gamma_{\rho\mu}^\alpha \Gamma_{\sigma\nu}^\beta R_{\alpha\beta}\\
    &=\Gamma_{\sigma\mu}^\alpha\Gamma_{\rho\alpha}^\beta R_{\beta\nu}
    + \Gamma_{\sigma\nu}^\alpha\Gamma_{\rho\alpha}^\beta R_{\mu\beta}
    + \cancel{\partial_\rho \Gamma_{\sigma\mu}^\alpha R_{\alpha\nu}}
    + \cancel{\partial_\rho\Gamma_{\sigma\nu}^\alpha R_{\mu\alpha}}
    + \Gamma_{\rho\sigma}^\alpha \Big(- \partial_\alpha R_{\mu\nu}
    + \Gamma_{\alpha\mu}^\beta R_{\beta\nu}
    + \Gamma_{\alpha\nu}^\beta R_{\mu\beta}\Big) \\
    &\quad 
    - \Gamma_{\sigma\mu}^\alpha \partial_\rho R_{\alpha\nu}
    - \cancel{\partial_\rho \Gamma_{\sigma\mu}^\alpha R_{\alpha\nu}}
    - \Gamma_{\sigma\nu}^\alpha \partial_\rho R_{\mu\alpha}
    - \cancel{\partial_\rho \Gamma_{\sigma\nu}^\alpha R_{\mu\alpha}}
    + \partial_\rho \nabla_\sigma R_{\mu\nu} 
    + 2 \Gamma_{\rho\mu}^\alpha \Gamma_{\sigma\nu}^\beta R_{\alpha\beta}\\
    &=\partial_\rho \nabla_\sigma R_{\mu\nu}
    - \Gamma_{\rho\sigma}^\alpha \nabla_\alpha R_{\mu\nu}
    + \Gamma_{\rho\alpha}^\beta \Gamma_{\sigma\mu}^\alpha R_{\beta\nu}
    + \Gamma_{\rho\alpha}^\beta \Gamma_{\sigma\nu}^\alpha R_{\mu\beta}
    - \Gamma_{\sigma\mu}^\alpha \partial_\rho R_{\alpha\nu}
    - \Gamma_{\sigma\nu}^\alpha \partial_\rho R_{\mu\alpha}
    + 2 \Gamma_{\rho\mu}^\alpha \Gamma_{\sigma\nu}^\beta R_{\alpha\beta}
\end{align*}
Now, since we will compute $g^{\rho\sigma}D_{\mu\nu\rho\sigma}$ we can (although is is incorrect before contraction, hence the ``$\approx$"\footnote{Here ``$\approx$" is thus the equivalence relation \textit{have equal symmetric parts}. So everything is rigorous, and in particular $X_{\rho\sigma} \approx Y_{\rho\sigma} \implies g^{\mu\nu}X_{\mu\nu}=g^{\mu\nu}Y_{\mu\nu}$ for $X_{\mu\nu}$ coefficients in any ring, including the enveloping algebra of $\mathbb{R}$, which is our case.})  rename $\rho \leftrightarrow \sigma$ in some contractions of $D$ and not others. Leading to:
\begin{align*}
    D_{\mu\nu\rho\sigma} 
    &\approx\partial_\rho \nabla_\sigma R_{\mu\nu}
    - \Gamma_{\rho\sigma}^\alpha \nabla_\alpha R_{\mu\nu}
    + \Gamma_{\sigma\alpha}^\beta \Gamma_{\rho\mu}^\alpha R_{\beta\nu}
    + \Gamma_{\sigma\beta}^\alpha \Gamma_{\rho\nu}^\beta R_{\mu\alpha}
    - \Gamma_{\rho\mu}^\alpha \partial_\sigma R_{\alpha\nu}
    - \Gamma_{\rho\nu}^\alpha \partial_\sigma R_{\mu\alpha}
    + 2 \Gamma_{\rho\mu}^\alpha \Gamma_{\sigma\nu}^\beta R_{\alpha\beta}\\
    &\approx\partial_\rho \nabla_\sigma R_{\mu\nu}
    - \Gamma_{\rho\sigma}^\alpha \nabla_\alpha R_{\mu\nu}
    + \Gamma_{\rho\mu}^\alpha \Big(
      \Gamma_{\sigma\alpha}^\beta  R_{\beta\nu}
    - \partial_\sigma R_{\alpha\nu}
    + \Gamma_{\sigma\nu}^\beta R_{\alpha\beta}
    \Big)
    + \Gamma_{\sigma\alpha}^\beta \Gamma_{\rho\nu}^\alpha R_{\mu\beta}
    - \Gamma_{\rho\nu}^\alpha \partial_\sigma R_{\mu\alpha}
    + \Gamma_{\sigma\mu}^\beta \Gamma_{\rho\nu}^\alpha R_{\beta\alpha}\\
    &\approx\partial_\rho \nabla_\sigma R_{\mu\nu}
    - \Gamma_{\rho\sigma}^\alpha \nabla_\alpha R_{\mu\nu}
    - \Gamma_{\rho\mu}^\alpha \nabla_\sigma R_{\alpha \nu}
    + \Gamma_{\rho\nu}^\alpha \Big( \Gamma_{\sigma\alpha}^\beta  R_{\mu\beta}
    - \partial_\sigma R_{\mu\alpha}
    + \Gamma_{\sigma\mu}^\beta R_{\beta\alpha} \Big)\\
    &\approx\partial_\rho \nabla_\sigma R_{\mu\nu}
    - \Gamma_{\rho\sigma}^\alpha \nabla_\alpha R_{\mu\nu}
    - \Gamma_{\rho\mu}^\alpha \nabla_\sigma R_{\alpha \nu}
    + \Gamma_{\rho\nu}^\alpha \nabla_\sigma R_{\mu\alpha}\\
    &\approx \nabla_\rho \nabla_\sigma R_{\mu\nu} = R_{\mu\nu;\rho;\sigma}
\end{align*}
So indeed the expressions stay the same.

Giving us:
$$|B_{\pm\pm}|  \leq \mathfrak{g} R_\mathrm{max}''$$

But even better, we can do the whole bound in a row, using a bit of linear algebra and the fact that
$$
C_{\pm\pm} = (2\alpha + \beta + 2 \gamma) A_{\pm\pm} + (4\gamma - \beta) B_{\pm\pm}
$$
Let us consider the function:
$$f_{\alpha\beta\gamma}: 
\begin{matrix}
    [-1, 1]^2 & \to & \mathbb{R}\\
    (a,b) & \mapsto & (2\alpha + \beta + 2 \gamma) a + (4\gamma - \beta) b
\end{matrix}$$
It is easy to see that $|C_{\pm\pm}| \leq \mathfrak{g} R_\mathrm{max}'' \cdot \mathrm{sup} f_{\alpha\beta\gamma}$.

Now, noticing that $f_{\alpha\beta\gamma}$ is a linear function (so constant gradient) restricted to the unit square, it is really easy to see that $\mathrm{sup}f_{\alpha\beta\gamma} = \mathrm{max} \big( |2\alpha + 6 \gamma|, |2\alpha + 2\beta - 2 \gamma|\big)$.\\
Leading to the bound:
\begin{equation}
    \boxed{\boxed{ |C_{\pm\pm}| \leq 2 \mathfrak{g} R_\mathrm{max}'' \cdot \mathrm{max}\Big( |\alpha + 3 \gamma|, |\alpha+\beta-\gamma|\Big)}}
\end{equation}






\end{document}